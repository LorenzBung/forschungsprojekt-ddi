\section{Umsetzung}
\label{sec:umsetzung}

Bevor die genauen Durchführungen der einzelnen Doppelstunden betrachtet werden, sind zunächst einige Vorbemerkungen zu machen.

Erstens waren die folgenden selbst gehaltenen Stunden nicht das erste Kennenlernen der Klasse.
Zuvor wurden bereits beide Hälften während Hospitationen bei Unterrichtsstunden zum Thema Python besucht und kennen gelernt.
Währenddessen konnten auch erste Beobachtungen bezüglich der sozialen und persönlichen Verhältnisse der Schülerinnen und Schüler gemacht werden, welche in die in \autoref{sec:konzeption} getroffene Konzeption des Unterrichts mit einfließen konnten.

Weiterhin war es nötig, dass die Klasse im eingeplanten Zeitraum eine Klassenarbeit schreiben musste, in welcher die in der vorhergehenden Unterrichtseinheit behandelten Inhalte (hauptsächlich Programmierung in Python) abgefragt werden würde.
Ein großer Nachteil davon war, dass die folgende Unterrichtseinheit durch eine Doppelstunde unterbrochen werden würde, in welcher die Arbeit geschrieben wurde.
Leider war es nicht möglich, den Termin anders zu legen, weswegen diese Unterbrechung mit in die Planung aufgenommen wurde.


\subsection{Erste Doppelstunde}
\label{subsec:doppelstunde-1}

Der Einstieg in den Unterricht erfolgte problemorientiert anhand eines Beispiels.
Gezeigt wurde eine Webseite, auf der die mit HTML und CSS gestaltete Eingabemaske eines Promillerechners zu sehen war.
Ein Ausschnitt der genannten Webseite ist in \autoref{fig:promillerechner} abgebildet.

\begin{figure}[h!]
	\centering
	\includegraphics[width=\textwidth]{media/Promillerechner.png}
	\caption{Webseite des Promillerechners}
	\label{fig:promillerechner}
\end{figure}

Bei Nutzung der Webseite wurde demonstriert, dass diese keinerlei Funktion hatte.
Ein großes Ziel der ersten Doppelstunde war, den aktuellen Vorwissensstand der Schülerinnen und Schüler zum Thema Webentwicklung herauszufinden.
Die präsentierte Webseite wurde so einerseits dazu genutzt, um den Schülerinnen und Schüler einen Ausblick auf die kommenden Stunden zu geben und ihnen klar zu machen, dass wir uns nun (wieder) mit dem Thema Webentwicklung beschäftigen.
Andererseits wurde damit auch eine Diskussion darüber angestoßen, aus was eine Webseite denn besteht und woran sie sich noch erinnern können.
An dieser beteiligten sich die Schülerinnen und Schüler sehr engagiert und es fielen auch Fachbegriffe wie HTML.
Auf Nachfrage wurde jedoch schnell ersichtlich, dass nur wenig Wissen über das Thema HTML vorhanden war.

Dies wurde als Überleitung zum ersten Arbeitsblatt genutzt, welches in \autoref{pdf:AB_HTML_CSS} zu sehen ist.
Das Übungsblatt sollte zunächst in Einzelarbeit bearbeitet werden, bei Schwierigkeiten konnten die Schülerinnen und Schüler jedoch selbstständig zur Partnerarbeit übergehen.
In der Besprechung des Blatts wurde schnell klar, dass die meisten Fragen nur durch Internetrecherche beantwortet werden konnten und viele der Lerninhalte aus der Einheit zu HTML nicht mehr präsent waren.
Mit CSS konnte nur ein Schüler etwas anfangen, die anderen hatten noch nie davon gehört.
Während dies für die Schülerinnen und Schüler frustrierend gewirkt haben kann, war es dennoch ein wichtiges Vorgehen.
Nun war klar, dass die kommenden Stunden nicht auf Dinge aufbauen konnten, die über ein Grundwissen von HTML hinausgingen und kein bzw. nur sehr grundlegendes CSS verwendet werden konnte.

Die anschließende Erklärung der Syntax von Javascript und ihren Aufbau als Programmiersprache verlief grundsätzlich produktiv.
Von den Schülerinnen und Schülern kamen nur wenige Fragen, jedoch wurde anschließend das in \autoref{pdf:Merkblatt_Variablen_if} stehende Merkblatt als sehr hilfreich erachtet.
Lediglich die für die kommenden Aufgaben notwendige Funktion \texttt{document.getElementById()} war schwierig zu verstehen.
Hier konnte die Verwendung durch zahlreiche Beispiele anhand der Nutzung der Funktion in der Konsole auf einer Webseite verdeutlicht werden.
Dennoch sollte dies ein Problem bleiben, welches dauerhaft bestehen blieb, was sich jedoch erst beim Feedback am Ende der Unterrichtseinheit herausstellte.

Die oben genannten Schwierigkeiten spiegelten sich auch in der Bearbeitung des folgenden Übungsblattes wider, welches sich in \autoref{pdf:AB_Variablen_if} finden lässt.
Viele Schülerinnen und Schüler wussten nicht weiter.
Resultierend daraus mussten zusätzliche Erklärungen und Beispiele eingeschoben sowie die Bearbeitungszeit für das Blatt nach oben angepasst werden.
Schlussendlich hatte es niemand geschafft, über Aufgabe 2 hinauszukommen.
Die Konsequenz daraus war, dass die Planung für die nächste Stunde angepasst werden musste, um den Schülerinnen und Schüler dort genügend Zeit für die restlichen Aufgaben zu ermöglichen, da in diesen das essentielle Konzept der if-Verzweigung eingeübt wurde.

Die Durchführung der Stunde in der anderen Hälfte der Klasse war insofern anders, als dass bei der Planung dieser auf die Vorerfahrungen aus der ersten Durchführung in der anderen Klassenhälfte zurückgegriffen werden konnte.
Beispielsweise war schon im Vorhinein klar, dass eine Wiederholung der Inhalte zu CSS nicht notwendig oder sogar verwirrend und frustrierend für die Schülerinnen und Schüler war.
Entsprechend wurde auch das Übungsblatt zur Vorwissensreaktivierung angepasst; diese Version findet sich in \autoref{pdf:AB_HTML}.
Außerdem konnte davon ausgegangen werden, dass die Schülerinnen und Schüler für die Bearbeitung der Aufgaben mehr Zeit als geplant benötigten und vertiefende Erklärungen zur Funktion \texttt{document.getElementById()} notwendig waren.
Auch um den Teil der Klasse auf einem ähnlichen Stand zu halten wurde also weniger Stoff in der Doppelstunde behandelt, dafür jedoch mit ausführlicheren Erklärungen.

Angenommen wurde dies von der Klasse gut.
Auch hier traten Schwierigkeiten bei der Verwendung der oben genannten Funktion auf, jedoch konnten diese die vertieften Erklärungen sowie zahlreiche Beispiele weitestgehend ausgeglichen werden.
Grundsätzlich wirkte dieser Teil der Klasse generell motivierter für das Fach und arbeitete konzentriert mit.
Einen Einfluss darauf könnte die durch das Fehlen einiger Schülerinnen und Schüler bedingte geringere Gruppengröße gehabt haben, was eine individuellere Förderung in den Arbeitsphasen ermöglichte.


\subsection{Zweite Doppelstunde}
\label{subsec:doppelstunde-2}

Leider wurde der Start des Unterrichts aufgrund technischer Schwierigkeiten verzögert.
Die Verbindung des Computers mit dem Whiteboard schlug fehl, was den Unterricht unmöglich machte.
Da die betreuende Lehrkraft wegen ihrer Tätigkeit in der Schulleitung leider von einem Kollegen benötigt wurde und sie daher nicht vor Ort war, um bei der Beseitigung des Problems zu helfen, dauerte es 10 Minuten, bis der Unterricht regulär starten konnte.

Der Einstieg in die zweite Stunde erfolgte mit einem Kahoot \cite{wang2020kahoot, dellos2015kahoot}, in dem es um die bereits erlernte Syntax von Javascript ging.
Dabei wurden auch mögliche Fehlvorstellungen bzw. Verwechslungen mit der Syntax von Python-Code aufgegriffen.
Das Quiz wurde von den Schülerinnen und Schülern sehr gut angenommen und schien die Motivation für die kommende Stunde deutlich zu steigern.
Ebenfalls konnten so Verständnisschwierigkeiten, die vom letzten Mal noch bestanden, beseitigt werden, beispielsweise die Verwendung von geschweiften Klammern in Javascript statt einem Doppelpunkt in Python.

Aufgrund der Schwierigkeiten, die bereits in der ersten Doppelstunde bei den Schülerinnen und Schülern aufgetreten waren, war es nicht möglich, sämtliche für die erste Stunde vorgesehenen Aufgaben zu bearbeiten und zu besprechen.
Wie bereits erwähnt kam niemand über Aufgabe 2 hinaus, während in der Planung von drei bearbeiteten Aufgaben ausgegangen war.
Aufgabe 3 war dabei so gestaltet, dass eine Lösung sowohl durch if-Verzweigungen, als auch durch die Nutzung einer Liste möglich war.
Da die Aufgabe jedoch noch nicht bearbeitet worden war, war auch die Nutzung dieser Aufgabe als Heranführung an das neue Thema nicht möglich.
Infolgedessen musste die Planung der Stunde verändert werden, so dass erst nachgeholt werden konnte, was in der letzten Stunde nicht mehr geschafft worden war.

Während der zusätzlich eingeplanten Arbeitszeit stellte sich heraus, dass die Schülerinnen und Schüler noch deutlich mehr Zeit benötigten als ursprünglich eingeplant.
In der zusätzlich eingeplanten Arbeitsphase von 25 Minuten schaffte es niemand, die noch offene Aufgabe 3 fertig zu bearbeiten, insbesondere deshalb, da auch einige Schülerinnen und Schüler noch mit Aufgabe 2 beschäftigt waren.
Nach Rücksprache mit der Klasse wurde daraufhin entschieden, die Doppelstunde als weitere Übungsstunde zu nutzen und das Thema Listen erst in der kommenden Stunde anzufangen.
Einfluss auf diese Entscheidung hatten auch der verzögerte Start sowie die Aussage der Klasse, mehr Zeit und eine zusätzliche Erklärung wären sehr hilfreich.

Da nun die gesamte Doppelstunde dazu genutzt werden konnte, die Übungsaufgaben 2 und 3 fertig zu bearbeiten, war es sämtlichen Schülerinnen und Schülern möglich, dieses Ziel am Ende der Stunde zu erreichen.
Sehr positiv war dabei, dass sich die Schülerinnen und Schüler untereinander geholfen haben und die schnelleren die langsameren unterstützt haben.
Auch in der abschließenden Besprechung der Aufgaben kamen insbesondere zur Bildergalerie zahlreiche verschiedene Lösungsansätze.
Dies regte unter den Schülerinnen und Schülern weitere Diskussionen darüber an, welche der vorgestellten Lösungen die eleganteste und beste sei.

Vor diesem Hintergrund stellte sich nun die Frage, wie die entsprechende Stunde in der zweiten Hälfte der Klasse durchzuführen wäre.
Einerseits war es ungünstig, die beiden Hälften auf unterschiedlichem Stand zu unterrichten, allerdings war es auch nicht sinnvoll, eine Doppelstunde ungenutzt zu lassen und Übungen zur Verfügung zu stellen, obwohl eigentlich neuer Stoff hätte behandelt werden können.
Schlussendlich fiel die Entscheidung darauf, ebenfalls eine weitere Übungsstunde in der anderen Klassenhälfte durchzuführen und das Thema Listen noch um eine Woche zu verschieben.
Dies geschah aus mehreren Gründen:
\begin{itemize}
	\item Die betreuende Lehrkraft erkrankte leider, was Unterstützungsmöglichkeiten in fachlicher, aber auch technischer Hinsicht einschränkte.
	\item Auch die zweite Hälfte der Klasse hatte Schwierigkeiten mit der Bearbeitung von Aufgabe 2 und 3 und war damit noch nicht fertig.
	\item Am Tag der Durchführung war ein Streik der Bahnen angekündigt und einige Schüler aus der zweiten Klassenhälfte hatten bereits erwähnt, nicht anwesend sein zu können.
	\item Es gab neue Schüler, welche bei der ersten Stunde nicht da gewesen waren, welche den Stoff nacharbeiten mussten.
	\item In der darauf folgenden Woche würde die Klassenarbeit zum Thema Python geschrieben werden, was voraussichtlich zu Ablenkungen bei den Schülerinnen und Schülern führen würde.
\end{itemize}
Schlussendlich waren tatsächlich nur drei der Schüler, welche auch in der ersten Doppelstunde da gewesen waren, wieder anwesend.
Zusätzlich dazu kamen zwei neue Schüler dazu, welche den Stoff nacharbeiten mussten.
Insgesamt hätte die Einführung neuen Stoffes mehr negative als positive Folgen gehabt.


\subsection{Dritte Doppelstunde}
\label{subsec:doppelstunde-3}

\subsection{Vierte Doppelstunde}
\label{subsec:doppelstunde-4}