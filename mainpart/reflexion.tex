\section{Reflexion}

Wie bereits in \autoref{subsec:doppelstunde-4} beschrieben nahmen die Schülerinnen und Schüler die Unterrichtseinheit weitestgehend positiv auf.
Die positiven Rückmeldungen waren unabhängig davon, ob es sich um leistungsstarke oder eher schwächere Schülerinnen bzw. Schüler handelte.
Als Verbesserungspotenzial wurden folgende Punkte genannt:

\begin{itemize}
	\item \textbf{Aufgabenschwierigkeit}: Einige Aufgaben, insbesondere die Wiederholungsaufgaben zu HTML und CSS sowie die Bildergalerie (Aufgabe 3, siehe \autoref{pdf:ab_variablen_if}) wurden nahezu einstimmig als zu schwer eingeschätzt.
	In der Konsequenz verloren insbesondere die schwächeren Schülerinnen und Schüler die Motivation und sahen keinen Sinn darin, sich überhaupt anzustrengen.
	Zum Ende der Unterrichtseinheit hin hat sich dies jedoch gebessert, da die Lerngeschwindigkeit der Schülerinnen und Schüler besser beurteilt werden konnte und die Aufgaben dementsprechend konzipiert wurden.

	\item \textbf{Herangehensweise}: Die in \autoref{subsubsec:lernstoff-gliederung} diskutierten Hintergründe zum gewählten Ansatz wurden von manchen Schülerinnen und Schülern negativ beurteilt.
	Der Grund hierfür war, dass viele Funktionen der Programmiersprache hingenommen werden mussten, ohne dass ihre Bedeutung oder Funktionsweise klar wurde.
	Dies führte verständlicherweise zu Verwirrung und Unsicherheit, wie diese Funktionen zu verwenden seien.
	Vermutlich wäre dies nicht der Fall gewesen, wenn der zweite Ansatz gewählt worden wäre - eine weniger praxisorientierte Herangehensweise, die jedoch klassischen Programmierkursen ähnelt und starke Parallelen zu Python aufweisen würde.
	Eine Möglichkeit, um diese Problematik zu vermeiden, wäre ein größerer zeitlicher Abstand zwischen den beiden Programmiersprachen.
	So würde die Einheit zu Javascript gleichzeitig auch die schon bekannten Herangehensweisen von Python wieder aufgreifen.
\end{itemize}

In Bezug auf die Leitfrage gab es zahlreiche positive, aber auch einige negative Effekte, die durch die Teilung der Klasse zustande kamen.
Auf Seite der Nachteile ist insbesondere der stark erhöhte Ressourcenbedarf einer Klassenteilung zu nennen.
Neben dem zusätzlichen zeitlichen Budget, für welches Lehrkräfte aufkommen müssen, sind selbstverständlich auch Raumbedarf und organisatorischer Mehraufwand zu nennen.
In diesem Fall war dies jedoch kein Hindernis, da die entsprechenden Ressourcen an der Schule zur Verfügung gestellt werden konnten.
Ein weiteres Argument gegen die Klassenteilung besteht darin, dass es innerhalb der Klassengemeinschaft zu einem ``Kampf'' um die begehrten Zeitslots gab.
Für eine der beiden Hälften lag der Unterricht immer so, dass sie eine Hohlstunde hatten, während dies bei der anderen Klassenhälfte nicht der Fall war.
Diesem Effekt ließe sich durch geeignete stundenplanerische Maßnahmen vorbeugen, jedoch ist es selbstverständlich schwierig bis unmöglich, eine für alle passende Lösung zu finden.

Zu den positiven Aspekten zählt, dass die Lehrkraft quasi nahezu dieselbe Stunde doppelt halten kann.
Dies führt dazu, dass lediglich kleine Anpassungen in der Planung der zweiten Stunde nötig sind und so viel Zeit gespart werden kann, welche beispielsweise in eine Verbesserung der Unterrichtsqualität investiert werden kann.

Zusätzlich ist es aufgrund der Erfahrungen in der ersten Doppelstunde möglich, Anpassungen an der Stunde vorzunehmen.
Damit können Schwierigkeiten bei der zweiten Durchführung verhindert und mögliche Verbesserungen integriert werden.
So war es beispielsweise in der ersten Woche möglich, nach den Erfahrungen mit der Wiederholung zum Thema HTML und CSS und den damit verbundenen Schwierigkeiten gegenzusteuern und in der zweiten Stunde den CSS-Teil vollständig wegzulassen.

Ein weiteres Argument ist, dass aufgrund der geringeren Gruppengröße deutlich mehr Möglichkeiten für die individuelle Förderung der Schülerinnen und Schüler zur Verfügung stehen.
Es war beispielsweise möglich, einer Schülerin 10 Minuten ununterbrochen zu helfen und eine erneute Erklärung in einer 1:1-Situation zu geben, was bei einer größeren Lerngruppe nahezu unmöglich gewesen wäre.
Davon profitieren nicht nur die Lernenden: es lässt sich so vermeiden, dass es zu großen Lernstandsdifferenzen innerhalb der Klasse kommt und reduziert somit den Differenzierungsbedarf, da Schwächen besser individuell ausgeglichen werden können.

Zusammenfassend kann festgehalten werden, dass die Vorteile bei einer Teilung der Klasse in zwei Hälften nahezu immer überwiegen, wenn an der Schule die nötigen Ressourcen zur Verfügung gestellt werden können.
Lediglich der organisatorische Mehraufwand stellt ein Problem dar, welches teilweise nicht ausgeglichen werden kann und eine zusätzliche Belastung an das Kollegium stellt.