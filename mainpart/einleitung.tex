\section{Einleitung}

Planung, Durchführung und Erfolg von schulischem Unterricht hängen von zahlreichen Faktoren ab.
Häufig genannt wird in Diskussionen dabei der Einfluss der Klassengrößen auf die Unterrichtsqualität und die Lernergebnisse der Schülerinnen und Schüler.
Sowohl weltweit \cite{hattie2008visible} als auch innerhalb Deutschlands \cite[S.~13f.]{wossmann2005kleinere} sind die Effekte auf den Bildungserfolg jedoch so gering, dass sie vernachlässigbar scheinen.
Dennoch wird die Klassenstärke als einer der häufigsten Gründe für die subjektive Belastung im Lehrkraftberuf angegeben \cite{bauer2003burn}.
Bei Aufteilung der Schülerinnen und Schüler einer Klasse auf zwei Hälften kommt es notwendigerweise zu einer Reduktion der Klassenstärke.
Aufgrund dessen ist das Ziel dieser Arbeit die Erforschung der Einflüsse von Klassenteilungen auf die Planung und Durchführung des Unterrichts sowie die Diskussion der wahrgenommenen Vor- und Nachteile dieser Vorgehensweise aus Perspektive der Lehrkraft.
Im Vordergrund steht dabei die Frage, welche didaktischen Auswirkungen die Teilung der Klasse nach sich zieht.

Zu den Tätigkeiten im Rahmen des Schulpraxissemesters zählen unter anderem die ``Planung und Durchführung [sowie] Analyse und Reflexion von Unterricht'' \cite[§7]{handreichung-praxissemester}.
Während des an der Richard-Fehrenbach-Gewerbeschule Freiburg absolvierten Praktikums kam es zu einer geplanten Klassenteilung im Kontext des Informatik-Unterrichts einer Jahrgangsstufe 2 mit Profilfach Technik und Management.
Somit bot es sich an, die in der Handreichung genannten Tätigkeiten mit den obigen Zielen zu verknüpfen und während der Durchführung einer vollständigen Unterrichtseinheit zu erproben.

Es handelte sich dabei an eine Reihe von insgesamt 4 Doppelstunden pro Klassenhälfte zum Thema Webentwicklung mit Javascript.
Im Folgenden werden zunächst Überlegungen bei der Konzeption der Stunden geschildert, gefolgt von einer Darstellung der Durchführung, dabei aufgetretener Probleme sowie der Erfahrungen, die dabei gewonnen werden konnten.
Diese werden anschließend reflektiert und es wird diskutiert, inwiefern die Ergebnisse in Hinblick auf die Forschungsfrage interpretiert werden können.