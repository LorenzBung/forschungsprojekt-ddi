\section{Konzeption}
\label{sec:konzeption}

\subsection{Voraussetzungsanalyse}
\label{sec:voraussetzungsanalyse}

Die in dieser Arbeit durchgeführte Voraussetzungsanalyse orientiert sich am von Paul Heimann begründeten Berliner Modell \cite[S.~41--70]{arnold2015}.
Darin spielen sowohl die anthropogenen als auch die sozio-kulturellen Voraussetzungen der Lerngruppe eine große Rolle.
Im Folgenden wird daher auf diese beiden Bedingungen genauer eingegangen, mögliche Auswirkungen auf die Lerngruppe genannt und daraus resultierende Konsequenzen auf die Planung des Unterrichts diskutiert.

\subsubsection{Anthropogene Voraussetzungen}
\label{sec:anthropogene-voraussetzungen}

Bei der Analyse der anthropogenen Voraussetzungen der Lerngruppe werden Faktoren berücksichtigt, welche die Schülerinnen und Schüler in die Unterrichtssituation mitbringen.
Fragen, die diese Analyse zu beantworten sucht, können unter Anderem die folgenden sein:
\begin{itemize}
	\item Welches Vorwissen bringt die Lerngruppe zum Thema der Unterrichtsstunde mit?
	\item Wie sind die sozialen Vorbedingungen, die innerhalb der Klasse herrschen?
	\item Wie sind Selbstwahrnehmung und Interesse der Schülerinnen und Schüler in Bezug auf das Fach, insbesondere zum aktuellen Thema?
\end{itemize}

Die im Rahmen dieser Arbeit durchgeführte Unterrichtseinheit fand in der zweiten Jahrgangsstufe eines Technischen Gymnasiums mit dem Profilfach ``Technik und Management'' statt.
Viele der Schülerinnen und Schüler haben zuvor eine Realschule besucht und dort die Mittlere Reife erhalten.
Das technische Gymnasium bietet ihnen somit die Möglichkeit, statt einer Ausbildung einen höheren Bildungsabschluss zu erlangen.
Für viele der Schülerinnen und Schüler ist die Perspektive nach dem Abitur ein Studium.
Unter Anderem aus diesem Grund brachten die meisten Lernenden eine vergleichsweise hohe Eigenmotivation mit.

Die Wahl der Schule und des Profilfaches spricht dafür, dass der Großteil der Lerngruppe ein hohes Interesse an technisch-mathematischen Themenbereichen mitbringt.
In Hospitationen im Unterricht mit der Klasse hat sich durch Gespräche und Beobachtungen jedoch herauskristallisiert, dass Informatik und informatische Inhalte von Teilen der Klasse als langweilig oder irrelevant empfunden wird.
Andere Schülerinnen und Schüler brachten bereits Einiges an Vorwissen mit und hatten auch Spaß an der Auseinandersetzung mit Problemstellungen aus dem Informatikunterricht.

Ein großes Problem vor der Durchführung der Unterrichtseinheit war die fehlende Klarheit über das Vorwissen der Schülerinnen und Schüler zum Thema ``Webentwicklung''.
Auch die betreuende Lehrkraft konnte diesbezüglich keine Angaben machen, da diese die Klasse erst in der Jahrgangsstufe übernommen hat, nicht jedoch in der Eingangsklasse.
Eben dann steht allerdings das Thema ``HTML und CSS'' im Bildungsplan \cite[BPE~2]{bildungsplan-tg-informatik}, was essentielles Vorwissen für die kommende Unterrichtseinheit ist.
Diese Frage konnte bis zur ersten Stunde auch nicht mehr geklärt werden, was die Planung dieser ersten Stunde erheblich erschwerte.
Klar war aber, dass alle Schülerinnen und Schüler Erfahrung im Bereich der Programmierung mit Python mitbrachten.
Dies war das Thema, mit dem sich die Klasse bis dahin beschäftigt hatte; es lag also auch noch keine große zeitliche Distanz dazwischen.

Insgesamt kann festgehalten werden, dass die Lerngruppe in Bezug auf ihr Vorwissen zwar relativ homogen war, bezüglich der Motivation für das Fach und seine Lerninhalte jedoch große Heterogenität herrschte.
Dies spiegelte sich auch deutlich in der Lerngeschwindigkeit der verschiedenen Schülerinnen und Schüler wider und war bei der weiteren Planung der Unterrichtseinheit zu berücksichtigen.

\subsubsection{Sozio-kulturelle Voraussetzungen}
\label{sec:sozio-kulturelle-voraussetzungen}

Hinter der Frage nach den sozio-kulturellen Voraussetzungen stehen sowohl externe wie auch innere Faktoren, die das Unterrichtsgeschehen beeinflussen können. Ein Teil der dazu zählenden Punkte sind:
\begin{itemize}
	\item Wie viel Zeit steht für den Unterricht zur Verfügung?
	\item Handelt es sich um eine Einzel- oder eine Doppelstunde?
	\item In welchen Räumlichkeiten findet die Stunde statt?
	\item Genügt die vorhandene technische Ausstattung dem geplanten Unterricht?
\end{itemize}
